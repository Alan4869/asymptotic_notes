\chapter{线性微分方程的近似解}
这一章和后面的非线性近似解、积分的渐近方法共同被称作是局部分析理论。
研究的是方程解在一个特定范围内的性质,而不是整体的性质。

这一章主要线性微分方程的相关理论,下一章主要讲非线性微分方程的结果。
\section{齐次线性微分方程奇点的分类}

这一节的内容可以参考数理方法中关于方程奇点和级数法的部分。

一个线性齐次微分方程可以写成如下的形式
$$y^{(n)}+p_{n-1}(x)y^{(n-1)}+\cdots +p_{1}(x)y'+p_0(x)=0$$
等式左边简记做$L_ny=y^{(n)}+\sum_{i=0}^{n-1}p_i(x)y^{(i)}$
\begin{definition}
    函数在某点的领域范围内可以展开成泰勒级数,则称
    函数在这一点处解析
\end{definition}
\begin{example}

    1. $y=e^x$在$(-\infty,+\infty)$上解析;2. $y=e^{\frac{1}{x}}$在0处不解析
\end{example}
\begin{proof}
    对于 $y=e^x$因为$y=\sum_{i=0}^{\infty}\frac{x^n}{n!}$

    对于$y=e^{\frac{1}{x}}$在0处展开成级数后,系数均为0,与原函数不相等;
    亦或者说展开的级数中含有$x$的负幂次项。
\end{proof}

\begin{definition}
    若方程$L_ny=0$的每一项系数在该点处解析,则该点是方程的常点
\end{definition}

对于一般的点,只要$|x|\lneq \infty$,都是比较容易判断的,但要判断$x=\infty$是不是方程的奇点时,应当要做变换$x=\frac{1}{t}$
用$t$做新的自变量,判断$t=0$是不是方程的常点。若是,则$x=\infty$是方程的常点;若不是,则$x=\infty$是方程的奇点

数学家Fuchs证明了如下结论
\begin{theorem}
    线性齐次微分方程在常点附近的解都是解析的,都能够展开成幂级数的形式。
    该幂级数的收敛半径至少时该常点到最近的奇点的距离,可能会比这个距离长,也可能不会。
    方程解的奇点一定是系数的奇点,但系数的奇点不一定就是解的奇点。
\end{theorem}
这里,我们略去定理的证明。
\begin{example}
    $(x^2+1)y'+2xy=0$的通解为$y=c(1+x^2)^{-1}$。解在零点处解析,但幂级数的收敛半径为1。
    把方程写成标准型$y'+\frac{2x}{x^2+1}y=0$显然,在$x=0$处系数是解析的因而解也应当是解析的;
    在复平面上,取$z=i$或$z=-i$这两点都是系数的奇点,因而幂级数形式的解收敛半径至少为1。
\end{example}
\begin{definition}
    若$x=x_0$不是方程的常点,但是$(x-x_0)^{n-i}p_i(x)$都是解析的,那么称$x=x_0$是方程的正则奇点(Regular Singular Points,RSP)。
\end{definition}
\begin{definition}
    若$x=x_0$既不是方程的常点,也不是方程的正则奇点,则称$x=x_0$是方程的非正则奇点(Irregular Singular Points,ISP)
\end{definition}
对于方程的正则奇点$x=x_0$,做变换$q_i(x)=(x-x_0)^{n-i}p_i(x)$。将此带回方程,会得到一个类似欧拉方程的方程。解当中可能会出现对数一样的表达式
在奇点附近的解的形式一般会是$y_1=(x-x_0)^{\alpha}A(x)$
或$y=(x-x_0)^{\alpha}B(x)ln(x-x_0)+C(x)(x-x_0)^{\beta}$。
更一般的表示成$$y=(x-x_o)^{\gamma}\sum_{i=0}^{n-1}[ln(x-x_0)]^iA_i(x)$$
上面的$A_i(x)$、$B(x)$、$C(x)$均是解析函数。

对于非正则的奇点,解在其上会表现出本征奇性,是一个本性奇点(参考数理方法孤立奇点的分类)

若存在一个变换,将原方程编程一个如下的微分方程组
$$y'(x)=a(x)y(x)+b(x)z(x)$$
$$z'(x)=c(x)y(x)+d(x)z(x)$$
且其系数在某点均没有奇异性,而原方程的系数在该点有奇异性,则在这一点处,方程的解无奇异性。

但这一结论用处不大,这样的变换比较难找。

\section{微分方程的级数法}

这一方法的大致思路是吧方程的解设成
$$y=\sum_{i=0}^{\infty}a_ix^i$$
或类似的的形式,将此带回微分方程当中,得到系数的递推表示,从而解出系数,得到解的表达
这一节的内容可以参考数理方法的相关章节。
\subsection{常点邻域内的级数法}
$x=x_0$是方程的常点,设其邻域内解可以写成$$y=\sum_{i=0}^{\infty}a_i(x-x_0)^i$$
的形式,代回方程,得到系数的递推公式,从而解出系数,得到方程的解。

\begin{example}
   暂时不想写例子 
\end{example}

\subsection{正则奇点领域内的级数解法}
对于方程
$$y''+\frac{p(x)}{x-x_0}y'+\frac{q(x)}{(x-x_0)^2}=0$$
$x=x_0$是方程的正则奇点。在此附近,方程的解具有一定的奇异性。Fuchs证明了这样的结论
\begin{theorem}
    二阶齐次微分方程在正则奇异点附近的解具有如下的形式
    \begin{equation}\label{fs_1}
        y_1=(x-x_0)^{\alpha_1}\sum_{n=0}^{\infty}a_n(x-x_0)^n
    \end{equation}
    \begin{equation}\label{fs_2}
        y_2=Ay_1\ln{(x-x_0)}+(x-x_0)^{\alpha_2}\sum_{n=0}^{\infty}b_n(x-x_0)^n
    \end{equation}
    两组线性无关的解要么都是\ref*{fs_1},要么是\ref*{fs_1}和\ref*{fs_2}
\end{theorem}
可见,方程一定有一个形如\ref*{fs_1}的解。
先不管会是哪一种情形,直接将方程中的$p(x)$和$q(x)$写成幂级数的形式,将解设成具有方程(\ref*{fs_1})的形式,并代入。
比较相应的系数,则可以得到一串的递推式。

令$a_0\neq 0$,那么自然是它的系数为0。即

称这一等式为判定方程,记作$P(\alpha)$。从这一方程中,可以解出$\alpha_1$和$\alpha_2$。假设$Re(\alpha_1)> Re(\alpha_2)$。

只要对所有的整数$n$,都有$P(\alpha+n)\neq 0$那么递推对可以进行下去。这要求$\alpha_1-\alpha_2$不能是整数。此时,微分方程的两个线性无关解都是\ref*{fs_1}的形式。
显然$P(\alpha_1+n)\neq 0$一定成立,因此$y_1=(x-x_0)^{\alpha_1}\sum_{n=0}^{\infty}a_n(x-x_0)^n$至少对应着判定方程中的大根。(也可能对应小的根,但一定对应大的那个)

若$P(\alpha)=0$是重根,即$\alpha_1=\alpha_2$。那么只有一个形如\ref*{fs_1}的解。另一个解会是形如\ref*{fs_2}。
实际上$$y_2=\frac{\partial y_1(\alpha,x)}{\partial \alpha}|_{\alpha=\alpha_1}$$。
证明的方法也很简单,将$y_1(\alpha,x)$代回原方程,两边对$\alpha$求导,令$\alpha=\alpha_1$,即可证得该结论。

若$\alpha_1-\alpha_2$是整数,设$\alpha_1=\alpha_2+N$。如果$a_N$的递推式右边也为0,那么$a_N$是任意的常数,原微分方程有两个形如\ref*{fs_1}的解。
如果$a_N$的递推式右边不为0,则原微分方程有一个形如\ref*{fs_2}的解。可以这样构造:将$\frac{\partial y_1(\alpha,x)}{\partial \alpha}|_{\alpha=\alpha_1}$代回原方程,得到新的一个方程。
寻找新方程的形如\ref*{fs_1}的解,设$ y^*=(x-x_0)^{\alpha_2}\sum_{n=0}^{\infty}a_n(x-x_0)^n$,那么$y_2=y^*-\frac{\partial y_1(\alpha,x)}{\partial \alpha}|_{\alpha=\alpha_1}$

当然,如果知道有一解形如\ref*{fs_2},则可以将\ref*{fs_2}带入原方程,也可以求得这个形式的解

\subsection{非正则奇点邻域内的渐近方法}