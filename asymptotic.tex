\documentclass[10pt, a4paper]{book}
\usepackage[left=1.50cm,right=1.50cm,top=1.5cm,bottom=1.5cm]{geometry}
\usepackage{amsmath, amsthm, amssymb, bm, graphicx, mathrsfs}
\usepackage[UTF8]{ctex}
\usepackage[colorlinks,
linkcolor = blue,		
anchorcolor = blue,
urlcolor = blue,		
citecolor = black,		
]{hyperref}




% \linespread{1.5}
\newtheorem{theorem}{定理}[section]
\newtheorem{definition}{定义}[section]
\newtheorem{lemma}[theorem]{引理}
\newtheorem{corollary}[theorem]{推论}
\newtheorem{example}{例}[section]
\newtheorem{proposition}[theorem]{命题}


\newenvironment{solution}{\par\noindent\textbf{解答. }}{\\\par}

\begin{document}

%\input{titlepage.tex}

\begin{titlepage}
    \vspace*{1.75\baselineskip}
    \begin{flushleft}
        \Huge{渐近法与摄动理论}\\
    \end{flushleft}
    %\vspace{0.2\baselineskip} % Whitespace below the title
	\rule{18cm}{0.4pt}\vspace*{-\baselineskip}\vspace{3.2pt}\\ % Thin horizontal rule
	\rule{18cm}{1.6pt} % Thick horizontal rule
	
    \vspace{15cm} % Whitespace after the title block
    \begin{flushleft}
       \large  红叶西江
    \end{flushleft}
    
        
\end{titlepage}

\pagenumbering{roman}
\setcounter{page}{1}

\begin{center}
    \Huge\textbf{前言}
\end{center}~\

这门课主要讲授如何近似求解常微分方程、近似求积分或者是级数等无法精确求解的东西。
开课老师为数学科学学院的鲁汪涛老师,中文授课,英文讲义和英文试卷。

成绩占比如下:作业30 \% ,课堂表现10\% ,最后的考试60\%。考试难度与作业难度相当,且安排在最后一节课上进行随堂考试,不做另外安排。

教材为《Advanced Mathematical Methods for Scientists and Engineers》中的第一卷《Asymptotic Methods and Perturbation Theory》。
课本的每一章开头,均有一段福尔摩斯的原话。

课程内容主要强调微分方程等连续的对象,对差分方程这样离散的的对象处理得比较简单或干脆略去。主要讲授课本的以下几章:

第一章主要复习常微分方程中的相关概念。第三章和第四章讲常微分方程的近似求解,分别对应线性和非线性两种情形。
第六章讲积分的渐近展开。第七章摄动级数,第八章级数求和的方法。第九章边界层理论,第十章WKB近似。(这份笔记的章节编号不会与书本编号相同)

其中,重点是第3,6,7,9,10五部分。

至少先修微积分和常微分方程这两门课,不论是一般的工科版还是专业的数院版。

%力学系和物理系的同学可以看一下这本书的边界层理论和WKB近似
\begin{flushright}
    \begin{tabular}{c}
        红叶西江\\
        \today
    \end{tabular}
\end{flushright}

\newpage
\pagenumbering{Roman}
\setcounter{page}{1}

\tableofcontents

\newpage
\setcounter{page}{1}
\pagenumbering{arabic}


\chapter{常微分方程}
这一章主要重温一下常微分方程的相关结论。

对于一个一般的线性微分方程(linear differential equation)可以写成
\[Ly=f(x)\]
\[L=p_0(x)+p_1(x)\frac{d}{dx}+……+p_{n-1}\frac{d^{n-1}}{dx^{n-1}}+\frac{d^n}{dx^n}\]
若算子$L$对于y不是线性的,则方程是非线性的(nolinear)。
若$f(x)=0$,则称方程是齐次的(homogeneous);否则,就是非齐次的(inhomogeneous)
最高阶导数的阶数即为方程的阶数。一般n阶方程的通解(general solution)具有n个独立的积分常数。
\section{初值问题(IVP) 和边值问题(BVP)}

初值问题(IPV,Initial Value Problems)是指,
对于一个常微分方程,给出的定解条件均是同一点处不同阶的导数值
\begin{example}
    一维情形下的牛顿第二定律就是一个二阶常微分方程,若给定初始时刻的速度和位置,就构成一个初值问题
    \[mx''(t)=f(t)\]
    \[x(0)=x_0\]
    \[x'(0)=v_0\]
\end{example}

边值问题(BVP,Boundary Value Problems)是指,
在不同点处,给定定解条件。边值问题的定解条件可以比初值问题更加的纷繁复杂,可以有类似偏微分方程中的多种边值条件。
\begin{example}
    下面定解问题就是一个边值问题
    \[y''=y^2+e^x\]
    \[y(0)=y'(1)=0\]
\end{example}

边值问题较初值问题可以更加复杂。初值问题可以根据方程和定解条件预先判断解的存在唯一性;而边值问题却无法这样处理
必须要先解出方程的通解,再考虑边界条件,从而确定解的存在性。而边界条件是可能导致问题无解的
\begin{example}
    \[x''+\pi^2x=0,      x\in[0,1]\]
    \[x(0)=0\]
    \[x(1)=1\]
    上述问题就不存在解。因为方程的通解为$x=A\sin(\pi t)+B\cos(\pi t)$
    ,而代入边界条件后得到的两个方程为\[B=0\]\[B=1\]显然这是矛盾的。因此这一定解问题无解
\end{example}
上述例子中若$x(1)=0$,则定解问题也可以有无数个解。

\section{齐次线性常微分方程}
对于一个一般的线性微分方程(linear differential equation)可以写成
\[Ly=f(x)\]
\[L=p_0(x)+p_1(x)\frac{d}{dx}+……+p_{n-1}\frac{d^{n-1}}{dx^{n-1}}+\frac{d^n}{dx^n}\]
它的通解一般是$n$个线性无关的解的线性组合,即可以表示成\[y=\sum_{i=1}^nc_iy_i(x)\]。
$c_i$是相应的积分常数,每一个$y_i$都是方程的解,但要求他们之间是线性无关的。

$n$个函数是否线性无关,可以用他们的Wronskian$W(x)$判断
\[W(x)=
\begin{vmatrix}
    y_1& y_2 & y_3 &  \cdots & y_n \\
    y_1'& y_2' & y_3' &  \cdots & y_n' \\
    
  \vdots & \vdots & \vdots &  & \vdots  \\
    y_1^{n-1}& y_2^{n-1} & y_3^{n-1} & \cdots & y_n^{n-1} 
  \end{vmatrix}
  \]
若函数族线性相关,则$W(x)$在一个区间上处处为0;若线性无关,则几乎处处不为0
但要注意的是,$W(x)\neq 0$是线性无关的充分不必要条件;$W(x)=0$是线性相关的必要不充分条件。
\begin{example}

    \begin{equation}
        x_1=\left\{
            \begin{aligned}
                0\quad -1\leq x\leq 0\\
                t^2 \quad 0\leq x\leq 1
            \end{aligned}
            \right .
    \end{equation}

    \begin{equation}
        x_2=\left\{
            \begin{aligned}
                t^2 \quad -1\leq x\leq 0\\
                0 \quad 0\leq x\leq 1
            \end{aligned}
            \right .
    \end{equation}

    这两个函数的朗斯基行列式为0。
    但按线性相关的定义,不存在非零常数$c_1$、$c_2$使得\[c_1x_1+c_2x_2=0\]因此这两个函数线性无关。
\end{example}

对于齐次线性微分方程呢个,它的$W(x)$满足下述定理
\begin{theorem}
    齐次线性微分方程的Wronskian$W(x)$满足一阶的微分方程
    \[W'(x)=-p_{n-1}(x)W(x)\]
\end{theorem}
\begin{proof}
    
\end{proof}

对于初值问题,朗斯基行列式可以用于检验问题是否适定(well-posed or ill-posed)。
所谓适定与否,即即方程的解是否存在且唯一。
若给定初值条件的点,是朗斯基行列式的零点,则方程将没有符合条件的解,或者有无穷多组解。

朗斯基行列式也可以用于判断某一区域是否是解的存在区域。只要在该区域$p_{n-1}(x)$不是奇异的,就存在解。

\begin{example}
    判断下列问题是否适定
    \[y''-y'\frac{1+x}{x}+\frac{y'}{x}=0\]
    \[y(0)=0\]\[y'(0)=2\]

方程的朗斯基行列式满足
    \[W'(x)=\frac{1+x}{x}W(x)\]
    解得\[W=xe^x\]
    x=0是$W(x)$的零点,因而问题是不适定的。
    实际上,$y=c_1e^x+c_2(1+x)$,代入初值条件
    \[c_1+c_2=0\]
    \[c_1+c_2=2\]
    显然方程组无解,因此问题是不适定的
\end{example}

但要注意的是,非零的朗斯基行列式不意味着问题适定。上述结论是一个必要但不充分的条件。
而对于朗斯基行列式为无穷的点,问题可能是适定的,也可能是不适定的。(见书本正文P10)


对于边值问题,其适定性的判断比较复杂。上述条件难以保证问题的适定性;但若不符合上述条件,则同样不适定

\section{齐次线性方程的解法}
参考常微分方程教材





\chapter{线性微分方程的近似解}
这一章和后面的非线性近似解、积分的渐近方法共同被称作是局部分析理论。
研究的是方程解在一个特定范围内的性质,而不是整体的性质。

这一章主要线性微分方程的相关理论,下一章主要讲非线性微分方程的结果。
\section{齐次线性微分方程奇点的分类}

这一节的内容可以参考数理方法中关于方程奇点和级数法的部分。

一个线性齐次微分方程可以写成如下的形式
$$y^{(n)}+p_{n-1}(x)y^{(n-1)}+\cdots +p_{1}(x)y'+p_0(x)=0$$
等式左边简记做$L_ny=y^{(n)}+\sum_{i=0}^{n-1}p_i(x)y^{(i)}$
\begin{definition}
    函数在某点的领域范围内可以展开成泰勒级数,则称
    函数在这一点处解析
\end{definition}
\begin{example}

    1. $y=e^x$在$(-\infty,+\infty)$上解析;2. $y=e^{\frac{1}{x}}$在0处不解析
\end{example}
\begin{proof}
    对于 $y=e^x$因为$y=\sum_{i=0}^{\infty}\frac{x^n}{n!}$

    对于$y=e^{\frac{1}{x}}$在0处展开成级数后,系数均为0,与原函数不相等;
    亦或者说展开的级数中含有$x$的负幂次项。
\end{proof}

\begin{definition}
    若方程$L_ny=0$的每一项系数在该点处解析,则该点是方程的常点
\end{definition}

对于一般的点,只要$|x|\lneq \infty$,都是比较容易判断的,但要判断$x=\infty$是不是方程的奇点时,应当要做变换$x=\frac{1}{t}$
用$t$做新的自变量,判断$t=0$是不是方程的常点。若是,则$x=\infty$是方程的常点;若不是,则$x=\infty$是方程的奇点

数学家Fuchs证明了如下结论
\begin{theorem}
    线性齐次微分方程在常点附近的解都是解析的,都能够展开成幂级数的形式。
    该幂级数的收敛半径至少时该常点到最近的奇点的距离,可能会比这个距离长,也可能不会。
    方程解的奇点一定是系数的奇点,但系数的奇点不一定就是解的奇点。
\end{theorem}
这里,我们略去定理的证明。
\begin{example}
    $(x^2+1)y'+2xy=0$的通解为$y=c(1+x^2)^{-1}$。解在零点处解析,但幂级数的收敛半径为1。
    把方程写成标准型$y'+\frac{2x}{x^2+1}y=0$显然,在$x=0$处系数是解析的因而解也应当是解析的;
    在复平面上,取$z=i$或$z=-i$这两点都是系数的奇点,因而幂级数形式的解收敛半径至少为1。
\end{example}
\begin{definition}
    若$x=x_0$不是方程的常点,但是$(x-x_0)^{n-i}p_i(x)$都是解析的,那么称$x=x_0$是方程的正则奇点(Regular Singular Points,RSP)。
\end{definition}
\begin{definition}
    若$x=x_0$既不是方程的常点,也不是方程的正则奇点,则称$x=x_0$是方程的非正则奇点(Irregular Singular Points,ISP)
\end{definition}
对于方程的正则奇点$x=x_0$,做变换$q_i(x)=(x-x_0)^{n-i}p_i(x)$。将此带回方程,会得到一个类似欧拉方程的方程。解当中可能会出现对数一样的表达式
在奇点附近的解的形式一般会是$y_1=(x-x_0)^{\alpha}A(x)$
或$y=(x-x_0)^{\alpha}B(x)ln(x-x_0)+C(x)(x-x_0)^{\beta}$。
更一般的表示成$$y=(x-x_o)^{\gamma}\sum_{i=0}^{n-1}[ln(x-x_0)]^iA_i(x)$$
上面的$A_i(x)$、$B(x)$、$C(x)$均是解析函数。

对于非正则的奇点,解在其上会表现出本征奇性,是一个本性奇点(参考数理方法孤立奇点的分类)

若存在一个变换,将原方程编程一个如下的微分方程组
$$y'(x)=a(x)y(x)+b(x)z(x)$$
$$z'(x)=c(x)y(x)+d(x)z(x)$$
且其系数在某点均没有奇异性,而原方程的系数在该点有奇异性,则在这一点处,方程的解无奇异性。

但这一结论用处不大,这样的变换比较难找。

\section{微分方程的级数法}

这一方法的大致思路是吧方程的解设成
$$y=\sum_{i=0}^{\infty}a_ix^i$$
或类似的的形式,将此带回微分方程当中,得到系数的递推表示,从而解出系数,得到解的表达
这一节的内容可以参考数理方法的相关章节。
\subsection{常点邻域内的级数法}
$x=x_0$是方程的常点,设其邻域内解可以写成$$y=\sum_{i=0}^{\infty}a_i(x-x_0)^i$$
的形式,代回方程,得到系数的递推公式,从而解出系数,得到方程的解。

\begin{example}
   暂时不想写例子 
\end{example}

\subsection{正则奇点领域内的级数解法}
对于方程
$$y''+\frac{p(x)}{x-x_0}y'+\frac{q(x)}{(x-x_0)^2}=0$$
$x=x_0$是方程的正则奇点。在此附近,方程的解具有一定的奇异性。Fuchs证明了这样的结论
\begin{theorem}
    二阶齐次微分方程在正则奇异点附近的解具有如下的形式
    \begin{equation}\label{fs_1}
        y_1=(x-x_0)^{\alpha_1}\sum_{n=0}^{\infty}a_n(x-x_0)^n
    \end{equation}
    \begin{equation}\label{fs_2}
        y_2=Ay_1\ln{(x-x_0)}+(x-x_0)^{\alpha_2}\sum_{n=0}^{\infty}b_n(x-x_0)^n
    \end{equation}
    两组线性无关的解要么都是\ref*{fs_1},要么是\ref*{fs_1}和\ref*{fs_2}
\end{theorem}
可见,方程一定有一个形如\ref*{fs_1}的解。
先不管会是哪一种情形,直接将方程中的$p(x)$和$q(x)$写成幂级数的形式,将解设成具有方程(\ref*{fs_1})的形式,并代入。
比较相应的系数,则可以得到一串的递推式。

令$a_0\neq 0$,那么自然是它的系数为0。即

称这一等式为判定方程,记作$P(\alpha)$。从这一方程中,可以解出$\alpha_1$和$\alpha_2$。假设$Re(\alpha_1)> Re(\alpha_2)$。

只要对所有的整数$n$,都有$P(\alpha+n)\neq 0$那么递推对可以进行下去。这要求$\alpha_1-\alpha_2$不能是整数。此时,微分方程的两个线性无关解都是\ref*{fs_1}的形式。
显然$P(\alpha_1+n)\neq 0$一定成立,因此$y_1=(x-x_0)^{\alpha_1}\sum_{n=0}^{\infty}a_n(x-x_0)^n$至少对应着判定方程中的大根。(也可能对应小的根,但一定对应大的那个)

若$P(\alpha)=0$是重根,即$\alpha_1=\alpha_2$。那么只有一个形如\ref*{fs_1}的解。另一个解会是形如\ref*{fs_2}。
实际上$$y_2=\frac{\partial y_1(\alpha,x)}{\partial \alpha}|_{\alpha=\alpha_1}$$。
证明的方法也很简单,将$y_1(\alpha,x)$代回原方程,两边对$\alpha$求导,令$\alpha=\alpha_1$,即可证得该结论。

若$\alpha_1-\alpha_2$是整数,设$\alpha_1=\alpha_2+N$。如果$a_N$的递推式右边也为0,那么$a_N$是任意的常数,原微分方程有两个形如\ref*{fs_1}的解。
如果$a_N$的递推式右边不为0,则原微分方程有一个形如\ref*{fs_2}的解。可以这样构造:将$\frac{\partial y_1(\alpha,x)}{\partial \alpha}|_{\alpha=\alpha_1}$代回原方程,得到新的一个方程。
寻找新方程的形如\ref*{fs_1}的解,设$ y^*=(x-x_0)^{\alpha_2}\sum_{n=0}^{\infty}a_n(x-x_0)^n$,那么$y_2=y^*-\frac{\partial y_1(\alpha,x)}{\partial \alpha}|_{\alpha=\alpha_1}$

当然,如果知道有一解形如\ref*{fs_2},则可以将\ref*{fs_2}带入原方程,也可以求得这个形式的解

\subsection{非正则奇点邻域内的渐近方法}
% \chapter{chapter02} 
% \section{ceshi}
% 测试
\end{document}