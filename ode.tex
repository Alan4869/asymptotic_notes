\chapter{常微分方程}
这一章主要重温一下常微分方程的相关结论。

对于一个一般的线性微分方程(linear differential equation)可以写成
\[Ly=f(x)\]
\[L=p_0(x)+p_1(x)\frac{d}{dx}+……+p_{n-1}\frac{d^{n-1}}{dx^{n-1}}+\frac{d^n}{dx^n}\]
若算子$L$对于y不是线性的,则方程是非线性的(nolinear)。
若$f(x)=0$,则称方程是齐次的(homogeneous);否则,就是非齐次的(inhomogeneous)
最高阶导数的阶数即为方程的阶数。一般n阶方程的通解(general solution)具有n个独立的积分常数。
\section{初值问题(IVP) 和边值问题(BVP)}

初值问题(IPV,Initial Value Problems)是指,
对于一个常微分方程,给出的定解条件均是同一点处不同阶的导数值
\begin{example}
    一维情形下的牛顿第二定律就是一个二阶常微分方程,若给定初始时刻的速度和位置,就构成一个初值问题
    \[mx''(t)=f(t)\]
    \[x(0)=x_0\]
    \[x'(0)=v_0\]
\end{example}

边值问题(BVP,Boundary Value Problems)是指,
在不同点处,给定定解条件。边值问题的定解条件可以比初值问题更加的纷繁复杂,可以有类似偏微分方程中的多种边值条件。
\begin{example}
    下面定解问题就是一个边值问题
    \[y''=y^2+e^x\]
    \[y(0)=y'(1)=0\]
\end{example}

边值问题较初值问题可以更加复杂。初值问题可以根据方程和定解条件预先判断解的存在唯一性;而边值问题却无法这样处理
必须要先解出方程的通解,再考虑边界条件,从而确定解的存在性。而边界条件是可能导致问题无解的
\begin{example}
    \[x''+\pi^2x=0,      x\in[0,1]\]
    \[x(0)=0\]
    \[x(1)=1\]
    上述问题就不存在解。因为方程的通解为$x=A\sin(\pi t)+B\cos(\pi t)$
    ,而代入边界条件后得到的两个方程为\[B=0\]\[B=1\]显然这是矛盾的。因此这一定解问题无解
\end{example}
上述例子中若$x(1)=0$,则定解问题也可以有无数个解。

\section{齐次线性常微分方程}
对于一个一般的线性微分方程(linear differential equation)可以写成
\[Ly=f(x)\]
\[L=p_0(x)+p_1(x)\frac{d}{dx}+……+p_{n-1}\frac{d^{n-1}}{dx^{n-1}}+\frac{d^n}{dx^n}\]
它的通解一般是$n$个线性无关的解的线性组合,即可以表示成\[y=\sum_{i=1}^nc_iy_i(x)\]。
$c_i$是相应的积分常数,每一个$y_i$都是方程的解,但要求他们之间是线性无关的。

$n$个函数是否线性无关,可以用他们的Wronskian$W(x)$判断
\[W(x)=
\begin{vmatrix}
    y_1& y_2 & y_3 &  \cdots & y_n \\
    y_1'& y_2' & y_3' &  \cdots & y_n' \\
    
  \vdots & \vdots & \vdots &  & \vdots  \\
    y_1^{n-1}& y_2^{n-1} & y_3^{n-1} & \cdots & y_n^{n-1} 
  \end{vmatrix}
  \]
若函数族线性相关,则$W(x)$在一个区间上处处为0;若线性无关,则几乎处处不为0
但要注意的是,$W(x)\neq 0$是线性无关的充分不必要条件;$W(x)=0$是线性相关的必要不充分条件。
\begin{example}

    \begin{equation}
        x_1=\left\{
            \begin{aligned}
                0\quad -1\leq x\leq 0\\
                t^2 \quad 0\leq x\leq 1
            \end{aligned}
            \right .
    \end{equation}

    \begin{equation}
        x_2=\left\{
            \begin{aligned}
                t^2 \quad -1\leq x\leq 0\\
                0 \quad 0\leq x\leq 1
            \end{aligned}
            \right .
    \end{equation}

    这两个函数的朗斯基行列式为0。
    但按线性相关的定义,不存在非零常数$c_1$、$c_2$使得\[c_1x_1+c_2x_2=0\]因此这两个函数线性无关。
\end{example}

对于齐次线性微分方程呢个,它的$W(x)$满足下述定理
\begin{theorem}
    齐次线性微分方程的Wronskian$W(x)$满足一阶的微分方程
    \[W'(x)=-p_{n-1}(x)W(x)\]
\end{theorem}
\begin{proof}
    
\end{proof}

对于初值问题,朗斯基行列式可以用于检验问题是否适定(well-posed or ill-posed)。
所谓适定与否,即即方程的解是否存在且唯一。
若给定初值条件的点,是朗斯基行列式的零点,则方程将没有符合条件的解,或者有无穷多组解。

朗斯基行列式也可以用于判断某一区域是否是解的存在区域。只要在该区域$p_{n-1}(x)$不是奇异的,就存在解。

\begin{example}
    判断下列问题是否适定
    \[y''-y'\frac{1+x}{x}+\frac{y'}{x}=0\]
    \[y(0)=0\]\[y'(0)=2\]

方程的朗斯基行列式满足
    \[W'(x)=\frac{1+x}{x}W(x)\]
    解得\[W=xe^x\]
    x=0是$W(x)$的零点,因而问题是不适定的。
    实际上,$y=c_1e^x+c_2(1+x)$,代入初值条件
    \[c_1+c_2=0\]
    \[c_1+c_2=2\]
    显然方程组无解,因此问题是不适定的
\end{example}

但要注意的是,非零的朗斯基行列式不意味着问题适定。上述结论是一个必要但不充分的条件。
而对于朗斯基行列式为无穷的点,问题可能是适定的,也可能是不适定的。(见书本正文P10)


对于边值问题,其适定性的判断比较复杂。上述条件难以保证问题的适定性;但若不符合上述条件,则同样不适定

\section{齐次线性方程的解法}
参考常微分方程教材




